\hypertarget{database}{%
\section{Database}\label{database}}

We use MariaDB, a popular fork of MySQL.

These instructions are for running the database in a Docker container
outside of \texttt{docker-compose}. See \href{../README.md}{the backend
README} for instructions for running a production/testing instance.

\hypertarget{instructions}{%
\subsection{Instructions}\label{instructions}}

To build the image and start it:

\begin{Shaded}
\begin{Highlighting}[]
\FunctionTok{make}\NormalTok{ maria}
\end{Highlighting}
\end{Shaded}

Alternatively, to start a local MariaDB server on WSL:

\begin{Shaded}
\begin{Highlighting}[]
\FunctionTok{make}\NormalTok{ wsl}
\end{Highlighting}
\end{Shaded}

To run the setup script against the database, which creates the
necessary tables and procedures:

\begin{Shaded}
\begin{Highlighting}[]
\FunctionTok{make}\NormalTok{ tables}
\end{Highlighting}
\end{Shaded}

To fill the tables with test data:

\begin{Shaded}
\begin{Highlighting}[]
\FunctionTok{make}\NormalTok{ data}
\end{Highlighting}
\end{Shaded}

\textbf{NB}: You might need to wait for a few seconds before executing
this command. Additionally, \emph{make sure} to have a \texttt{.env}
file in the backend folder.

To clean up the containers:

\begin{Shaded}
\begin{Highlighting}[]
\FunctionTok{make}\NormalTok{ clean}
\end{Highlighting}
\end{Shaded}

Not for Windows users, or others who run into trouble: If these
instructions do not work, try cloning the repo with LF line endings
instead of Windows' default CRLF -- or just look at the \url{Makefile}
and \href{setup.sh}{setup script} and adapt the commands to your needs.
